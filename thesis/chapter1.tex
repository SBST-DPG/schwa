\chapter{Introduction} \label{chap:intro}

\section*{}
We review the state of the art in Mining Software Repositories (MSR),
existing tools and propose a new method to predict defects based on data
extracted from repositories. Also we use this information from defect
prediction to improve the diagnostic accuracy from Crowbar, namely,
the Barinel algorithm.

\section{Context} \label{sec:context}
Software plays an important role for society and in our daily routine, since
we use applications to communicate, manage information, etc. We expect that
these applications behave correctly and we are easily frustrated when they
are defective. Development of software is not a simple task since developers
need to maintain complex code, test and manage expectations of stakeholders
by correctly interpreting requirements. It is estimated that fixing bugs
represent 90\% of development costs \cite{Servant1}.

There are tools that can help developers delivering high quality software,
by automatically reviewing code and analyzing their behaviour. Some of
these tools are Codacy\footnote{\url{https://codacy.com/}},
Crowbar\footnote{\url{http://crowbar.io/}} and
Codeclimate\footnote{\url{https://codeclimate.com/}}. The usage of revision
control systems such as Git, SVN and Mercurial, helps developers tracking
changes on Software and understanding the evolution of components. Tools are
important to developers since they automate and avoid repetitive tasks in
Software development.

With the growing usage of revision control systems, research in MSR evolved
in the last decade and involves the analysis of systems used to support
the development of software such as repositories, issue trackers and
mailing lists \cite{Hemmati2013}.

\section{Motivation and goals} \label{sec:goals}

Software repositories have hidden information that can be explored with
analytics and machine learning techniques to support defect prediction
models. The Barinel algorithm in Crowbar uses static estimations for
defect prediction and goodness of components
\cite{Abreu:2009:SMF:1747491.1747511}. Insights from software history could
substitute these estimations and improve the diagnostic accuracy with more
dynamic estimations.

Our goals are:

\begin{itemize}
\item Predict defects from Software repositories by learning what are the most
important features to analyze and create a prediction model based from existing
techniques;
\item Improve diagnostic accuracy of Barinel with defect prediction
 probabilities.
\end{itemize}

\section{Concepts and definitions} \label{sec:concepts}
Software quality is the degree to which the system meets requirements and
expectations of users. The main characteristics of product quality,
by ISO/IEC 25010, are portability, maintainability, security, reliability,
functional stability, performance efficiency, compatibility and
usability \cite{isoiec25010:2011}.

\subsection{Software testing glossary}
According to ISTQB \footnote{\url{http://istqb.org}}, the standard glossary
 used in software testing is:
\begin{description}
  \item[Error] \hfill \\
   A human action that produces an incorrect result;
  \item[Fault, defect, bug] \hfill \\
  Flaw in a component that causes the system to fail performing its required
   functions;
  \item[Failure] \hfill \\
  Deviation of the component from its expected result.
\end{description}

\subsection{Types of tests}

Testing is the process, consisting of static and dynamic activities, that
involves the planing, preparation and evaluation of software to check if
it meets the requirements and to detect defects.

\begin{description}
\item[Static testing] \hfill \\
Is the analysis of the system static representation such as source code and
documents, improving the internal quality.

\item[Dynamic testing] \hfill \\
It involves the execution of the system, observing its behavior under test
cases, improving external quality.
\end{description}

\subsection{Defect prediction}
Defect prediction consists in reliably predicting software defects using
information from code metrics, process metrics or previous
defects \cite{D'Ambros:2012:EDP:2318097.2318149}. Some approaches use
binary classification, asserting if a component is defective or not,
while others rank them by giving a score.

\subsection{Fault localization}
Fault localization consists in finding the component that caused the Software
to fail. Is one of the most time consuming activities in debugging. Considering
this, there is a high demand for making this process automatic, leading to the
development of techniques that makes this activity more effective
\cite{Wong09asurvey}.

\section{Problem statement}
Considering our goals, we want to answer the following research questions:

\begin{description}
\item[RQ1] \hfill \\
What features should we extract and analyze from Software Repositories to
predict defects?

\item[RQ2] \hfill \\
Can we improve Barinel fault localization technique from Crowbar with the
 results from defect prediction?
\end{description}

\section{Contributions}
This thesis makes the following contributions:

\begin{itemize}
\item A technique to parse and represent diffs from patches achieving method
granularity in Java;
\item A model to compute defect probabilities;
\item A framework for mining Software repositories and reporting analytics
in a graphical visualization;
\item A technique to learn the importance of tracked metrics;
\item A method to evaluate the gain of using defect probabilities in fault
 localization, namely the Barinel algorithm in Crowbar.
\end{itemize}

\section{Document overview}
\begin{description}
  \item[1. Introduction] \hfill \\
  An introduction to the context, goals and concepts of this thesis.

  \item[2. State of the art] \hfill \\
  Current state of the art techniques on Mining Software Repositories and
  Software Debugging are reviewed, along with example of existing tools.

  \item[3. A Technique to Estimate Defect Probabilities] \hfill \\
  It is presented the tool created to conduct this research and the
  methodologies used.

  \item[4. Experimental results] \hfill \\
  The experimental setup and results are presented in this chapter.

  \item[5. Discussion] \hfill \\
  A chapter dedicated to the discussion of findings about the results.
  Initial research questions are answered.

  \item[6. Conclusions and Further Work] \hfill \\
  The conclusions and satisfaction of goals are discussed, along with an
  overview of further work.

\end{description}
