\chapter{Discussion} \label{chap:discussion}

\section*{}
In this chapter it is discussed findings and conclusions relative to the initial
research questions.

\section{Features weight estimation}
The initial goal was finding a way of generalizing the weights of each features.
Although, since every software project is different, we found that it depends on
the project:

\begin{description}
\item[Features weights are different for each project] \hfill \\
In Schwa that is a project with only contributor, the weights for 50 and 100
commits were consistent: revisions is the most important feature. But, for Joda
Time with 100 commits, the most important was fixes.

\item[Noise on tracking fixes] \hfill \\
Authors and revisions are the only features that are measured with accuracy.
Fixes are tracked based on bug-fixing commits that have noise and have an impact
on defect predictions results, a problem that is discussed in MSR
research\cite{herzig-tr-2012}.

\item[Precision of Genetic Algorithms] \hfill \\
We represented individuals with 3 bits of precisions at the cost of performance.
With more computational power (e.g. cluster) we could increase precision to see
if we would find different results.

\item[Configurable features weights] \hfill \\
Since features could not be generalized, we introduced a new feature to Schwa:
a configuration file .schwa.yml, to allow developers to change the weights of
revisions, fixes and authors. With this, developers can run Schwa in learning
mode first and configure it with the weights learned.
\end{description}

\section{Diagnostic cost}
The results from diagnostic cost experiments indicate that we could not improve
the results of Crowbar but found an alternative way of estimating defect
probabilities in the Barinel technique:

\begin{description}
\item[Improvement of diagnostic results] \hfill \\
We could not find an example of Schwa improving the diagnostic results of
Crowbar. But, we must note that even with optimal defect predictions results
from Schwa, in some cases the diagnostic cost cannot be improved, as seen in
Joda Time.

\item[Importance of recently changed components] \hfill \\
In the first results from Joda Time we were getting worse results because faulty
components that had been recently changed, had low defect probability. By
modifying the TWR function with the Time Range parameter, when Schwa was used
in priors, it did not got worse results.

\item[Faster defect prediction results with Schwa] \hfill \\
Since the Barinel algorithm uses the MLE to estimate goodnesses and priors, this
process can take for example 2 hours in some cases. By using Schwa, we reduced
this phase to less than 1 minute.

\item[Computational power] \hfill \\
A cluster is better suited than a laptop to get results in a more convenient
time. Schwa is I/O intensive because it is parsing and extracting code from
commits. Crowbar have a substantial time complexity by running the MLE algorithm
and can benefit of faster CPUs.

\end{description}

\section{Threats to validity}
Regarding the experiments for estimating features weight, the usage of 3 bits
for representing the weights of individuals can limit the possibility of
searching better solutions. For the diagnostic cost, the results are just
from two projects that are open source.
