\chapter*{Abstract}
Finding and fixing software bugs is expensive and has a significant impact in Software development effort. Repositories have hidden predictive information about Software history that can be explored using analytics and machine learning techniques. A Software component can be a file, class or method in terms of granularity. Current research in Mining Software Repositories (MSR) is capable of ranking and listing faulty components at the file granularity. Crowbar is an automatic Software debugging tool that uses a technique named Barinel. Our goals are predicting Software defects with method granularity and improve Crowbar, by mining repositories. 

We have implemented a tool named Schwa, available for free on Github, that is capable of analyzing Git repositories. We are analyzing metrics such as revisions, fixes, authors and the time of commits to feed the prediction model. The analysis of time provides a method to ignore old components. Experimental results shown that for every Software repository, the predictive power of each metric is different. For example, in some projects revisions is more correlated with future defects and in others is fixes. The usage of defect predictions from Schwa in Crowbar reduced the amount of time necessary to rank faulty components. In the Joda Time project the time was reduced from one hour to less than a minute.

This thesis does the following contributions: a method to parse and represent diffs from patches with method granularity for Java; a model to compute defect probabilities; a framework for mining Software repositories; a technique to learn the importance of tracked metrics; a method to evaluate the gain of using defect probabilities in fault localization. 

\chapter*{Resumo}
Encontrar e corrigir bugs tem um grande custo e impacto no esforço em desenvolver Software. Os repositórios escondem informação preditiva sobre o histórico de Software que pode ser explorada recorrendo a técnicas de análise e de machine learning. Um componente de Software pode ser um ficheiro, classe ou método em termos de granularidade. A investigação atual de Mining Software Repositories (MSR) é capaz de classificar e listar componentes defeituosos com a granularidade ao nível do ficheiro. O Crowbar é uma ferramenta que faz depuração automática de Software e usa a técnica Barinel. Os nossos objetivos são prever defeitos em Software com granularidade até ao método e melhorar o Crowbar, ao extrair informação de repositórios.

Foi implementada uma ferramenta denominada de Schwa, disponível livremente no Github, que é capaz de analisar repositórios Git. Estamos a analisar métricas como as revisões, correções de bug, autores e o tempo dos commits para alimentar o modelo de previsão. A análise do tempo permite ignorar componentes mais antigos. Os resultados experimentais demonstraram  que para cada repositório de Software, o poder preditivo de cada métrica é diferente. Por exemplo, em alguns projetos o número de revisões está mais correlacionado com futuros defeitos e em outros é o número de correções de bugs. A utilização das previsões de defeito do Schwa no Crowbar reduziu o tempo necessário para classificar componentes faltosos. No projecto Joda Time o tempo foi reduzido de uma hora para menos de um minuto.

Esta tese faz as seguintes contribuições: um método para interpretar e representar diffs de patches com a granularidade ao método; um modelo para calcular probabilidades de defeito; uma framework para minar repositórios de Software; uma técnica para aprender a importância das métricas analisadas; um método para avaliar o ganho de usar as probabilidade de defeito em localização de falhas.